\documentclass{article}

\usepackage{amsmath, amsthm, amssymb, amsfonts}
\usepackage{thmtools}
\usepackage{graphicx}
\usepackage{setspace}
\usepackage{geometry}
\usepackage{float}
\usepackage{hyperref}
\usepackage[utf8]{inputenc}
\usepackage[english]{babel}
\usepackage{framed}
\usepackage[dvipsnames]{xcolor}
\usepackage{tcolorbox}

\colorlet{LightGray}{White!90!Periwinkle}
\colorlet{LightBlue}{Blue!10}
\colorlet{LightOrange}{Orange!15}
\colorlet{LightGreen}{Green!15}

\newcommand{\HRule}[1]{\rule{\linewidth}{#1}}

\declaretheoremstyle[name=Theorem,]{thmsty}
\declaretheorem[style=thmsty,numberwithin=section]{theorem}
\tcolorboxenvironment{theorem}{colback=LightGreen}

\declaretheoremstyle[name=Definition,]{prosty}
\declaretheorem[style=prosty,numberwithin=section]{definition}
\tcolorboxenvironment{definition}{colback=LightBlue}

\declaretheoremstyle[name=Principle,]{prcpsty}
\declaretheorem[style=prcpsty,numberwithin=section]{principle}
\tcolorboxenvironment{principle}{colback=LightGreen}


\setstretch{1.2}
\geometry{
    textheight=9in,
    textwidth=5.5in,
    top=1in,
    headheight=12pt,
    headsep=25pt,
    footskip=30pt
}

% ------------------------------------------------------------------------------

\begin{document}

% ------------------------------------------------------------------------------
% Cover Page and ToC
% ------------------------------------------------------------------------------

\title{ \normalsize \textsc{}
		\\ [2.0cm]
		\HRule{1.5pt} \\
		\LARGE \textbf{\uppercase{Math 427}
		\HRule{2.0pt} \\ [0.6cm] \LARGE{Subtitle} \vspace*{10\baselineskip}}
		}
\date{}
\author{\textbf{Author} \\ 
		Terry Qiu \\
		Urbana-Champaign \\
		Fall 2025}

\maketitle
\newpage

\tableofcontents
\newpage

% ------------------------------------------------------------------------------

\section{Integers and Permutations}

\subsection{Integers Modulo $n$}

\begin{definition}{Congruence modulo $n$}
    Let $a,b$ be integers. We say that $a$ is congruent to $b$ modulo $n$ if
    \[n|(a-b) \]
    where $n$ is some integer also. 
\end{definition}
Alternatively, notice that $a$ is congruent to $b$ modulo $n$ also means that 
$a$ has remainder $b$ when divided by $n$, $a=nk+b$.

\vspace{4mm}
Congruence as such defined a equivalence relation on $\mathbb{Z}$. In other words,
\[a\equiv b\pmod{n} \iff n|(a-b)\]
\begin{theorem}
    \[a\equiv b\pmod{n} \iff n|(a-b)\]
    This defines a valid equivalence relation, for $n\geq 2$.
\end{theorem}

\begin{proof}
    It suffices to show that $a\equiv b \pmod{n}$ is reflexive, transitive and symmetric.
    
    \vspace{4mm}
    Notice that for any $a\in \mathbb{Z}$ and any $n\geq 2$, we have 
    \[ n|(a-a)=n|0\]
    So $a\equiv a \pmod{n}$ for any $a$. If $n|(a-b)$ and $n|(b-c)$, then
    \[k_1n=a-b, \quad k_2n=b-c\]
    But then 
    \[n(k_1+k_2)= a-c\]
    So $n|(a-c)$, which implies $a\equiv c\pmod{n}$. For symmetry, notice that 
    \[n|(a-b) \implies k_1n=(a-b) \implies (-k_1)n =(b-a)\]
    So $n|(b-a)$.
    \vspace{4mm}
\end{proof}
\begin{theorem}
    If $a\in \mathbb{Z}$, then $[a]=[r]$ for some $0\leq r\leq n-1$, where $n\geq 2$.
\end{theorem}
\begin{proof}
    If $0\leq a\leq n-1$ then we are done. So, there are really two cases to consider
    $a\geq n$ or $a\leq -1$. Suppose that $a\geq n$, then
    \begin{align*}
        [x]=\{a | a\equiv x \text{ mod }n\}
    \end{align*}
    By the division algorithm, $x=nk+r$ for $0\leq r \leq n-1$. But this means that 
    \begin{align*}
        x\equiv r \text{ mod } n
    \end{align*}
    So in fact 
    \begin{align*}
        [x]=\{a | a\equiv r \text{ mod }n\}=[r]
    \end{align*}
    The same is true when $x\leq -1$.
\end{proof}
\begin{theorem}
$[a]=[b]$ if and only if $a\equiv b \pmod{n}$.
\end{theorem}
What this means is that to prove $a\equiv b \pmod{n}$ for some $n\geq 2$. It suffices 
to prove $[a] =[b]$ in $\mathbb{Z}_n$.
\\\\
For example, suppose we wanted to find the remainder of $4^{119}$ when divided by $7$.
It suffices to find the $0\leq r < 7$ such that $4^{119} \equiv r \pmod{7}$. But for any such $r$, we must have
\begin{align*}
    [4^{119}]= [r]
\end{align*}
We can work out $r$ by working with multiplication in $\mathbb{Z}_7$.\\\\
\textbf{Example:}
Prove that $a$ is divisible by $9$ if and only if the sum of its digits is divisible by $9$.
\begin{proof}
    Let $a$ be some arbitrary integer and consider it as its expansion in base $10$.
    Suppose without loss of generality that
    \begin{align*}
        a=d_nd_{n-1}...d_0
    \end{align*}
    Then
    \begin{align*}
        a=d_n\cdot10^n + ...+ d_0 \cdot 10^0
    \end{align*}
    It suffices to show that
    \begin{align*}
        a \equiv 0 \pmod{9}
    \end{align*}
    But this is true in integer mod $9$ if and only if 
    \begin{align*}
        [a]=[0]
    \end{align*}
    But then
    \begin{align*}
        [a]=[d_n\cdot10^n + ...+ d_0 \cdot 10^0]=[d_n][10^n]+...+[d_0]
    \end{align*}
    Notice that
    \begin{align*}
        10\equiv 1 \pmod{9}
        \implies 10^n \equiv 1^n \equiv 1\pmod{9}
    \end{align*}
    So, $[10^n]=[1]$ for any $n\in\mathbb{N}$. This means that
    \begin{align*}
        [a]=[d_n\cdot10^n + ...+ d_0 \cdot 10^0]=[d_n][1]+...+[d_0]= [d_n]+...+[d_0]
    \end{align*}
    So 
    \begin{align*}
        a \equiv d_n+d_{n-1}+...+d_0 \pmod{9}
    \end{align*}
    Now suppose that $a$ is divisible by $9$, then by transitivity the sum of the digits is divisible by $9$. On the otherhand, if the sum
    of the digits is divisible by $9$, then by transitivity $a$ is divisible by $9$ too. This is what we needed to show.
    
\end{proof}
\begin{definition}
    We call $[0]$ and $[1]$ the roots and unity of $\mathbb{Z}_n$ respectively.
\end{definition}
Suppose we are presented with the problem of solving congruence equations. For example, consider
in integers modulo $17$, the equation
\begin{align*}
    5x\equiv 2 \text{ mod } 17
\end{align*}
We want to find a $x$ such that $5x$ has remainder $2$ when divided by $17$. We can guess the solution by trial an error, but the better way is to 
consider the following problem. We ask whether $5$ has an inverse element.
\begin{definition}
    An inverse element of $[a]$ in the integers modulo $n$ is the class $[b]$ such
    that $[a][b]=[1]$.
\end{definition}
Why does this matter? If $5$ has an inverse then we can multiply both sides by that inverse. Suppose that the inverse 
is $[b]$. We have
\begin{align*}
    [5][b][x]&=[2][b] \\
    [x]&=[2][b]=[2b]
\end{align*}
If an inverse exists, then $[x]=[2b]$. Does there exists a solution to congruence relations of the sort 
\[ax\equiv b \text{ mod } n\]
when $a$ is non invertible? We have a solution if $d=\gcd(a,n)\mid b$ and we dont have a solution when $d=\gcd(a,n)\nmid b$.
We have the following theorem relating to whether an element is invertible in $\mathbb{Z}_n$.
\begin{theorem}
    $[a]$ is invertible in $\mathbb{Z}_n$ if and only if $\gcd(a,n)=1$.
\end{theorem}
\begin{proof}
    If $\gcd(a,n)=1$, then $1=ax+ny$ for some integers $x,y$. This means alternatively that $n|(1-ax)$ which means
    \[[ax]\equiv [1] \text{ mod } n\]
    So $[a]$ is invertible in particular by $[x]$. The reverse is also obvious.
\end{proof}
The following theorem is convenient for solving congruence relations.
\begin{theorem}{Chinese Remainder Theorem or the CRT}
    The equations
    \begin{align*}
        x \equiv s \text{ mod } m \quad x \equiv t \text{ mod } n
    \end{align*}
    has a solution $x$ when $m,n$ are relatively prime.
\end{theorem}
\begin{proof}
    Suppose $m,n$ are relatively prime. We can write $1$ as a linear combination of
    $m$ and $n$ by Bezout's identity.
    \begin{align*}
        1=mx+ny
    \end{align*}
    Let \begin{align*}
        x=mxt+nys
    \end{align*}
    Then
    \begin{align*}
        x-s = mxt+(ny-1)
    \end{align*}
    But $ny-1 =-mx$. So 
    \begin{align*}
        x-s = mxt-mx =mx(t-1)
    \end{align*}
    So $m|(x-s)$. This is what it means for $x$ to be congruent to $s$ modulo $m$.
    Similarly, 
    \begin{align*}
        x-t&=(mx-1)t+nys \\
        x-t &= ny(s-t)
    \end{align*}
    So $n|(x-t)$.

    
\end{proof}

% ------------------------------------------------------------------------------
% Reference and Cited Works
% ------------------------------------------------------------------------------

\bibliographystyle{IEEEtran}
\bibliography{References.bib}

% ------------------------------------------------------------------------------

\end{document}