\documentclass{article}

\usepackage{amsmath, amsthm, amssymb, amsfonts}
\usepackage{thmtools}
\usepackage{graphicx}
\usepackage{setspace}
\usepackage{geometry}
\usepackage{float}
\usepackage{hyperref}
\usepackage[utf8]{inputenc}
\usepackage[english]{babel}
\usepackage{framed}
\usepackage[dvipsnames]{xcolor}
\usepackage{tcolorbox}

\colorlet{LightGray}{White!90!Periwinkle}
\colorlet{LightBlue}{Blue!10}
\colorlet{LightOrange}{Orange!15}
\colorlet{LightGreen}{Green!15}

\newcommand{\HRule}[1]{\rule{\linewidth}{#1}}

\declaretheoremstyle[name=Theorem,]{thmsty}
\declaretheorem[style=thmsty,numberwithin=subsection]{theorem}
\tcolorboxenvironment{theorem}{colback=LightGreen}

\declaretheoremstyle[name=Definition,]{prosty}
\declaretheorem[style=prosty,numberwithin=subsection]{definition}
\tcolorboxenvironment{definition}{colback=LightBlue}

\declaretheoremstyle[name=Principle,]{prcpsty}
\declaretheorem[style=prcpsty,numberwithin=subsection]{principle}
\tcolorboxenvironment{principle}{colback=LightGreen}


\setstretch{1.2}
\geometry{
    textheight=9in,
    textwidth=5.5in,
    top=1in,
    headheight=12pt,
    headsep=25pt,
    footskip=30pt
}

% ------------------------------------------------------------------------------

\begin{document}

% ------------------------------------------------------------------------------
% Cover Page and ToC
% ------------------------------------------------------------------------------

\title{ \normalsize \textsc{}
		\\ [2.0cm]
		\HRule{1.5pt} \\
		\LARGE \textbf{\uppercase{Math 427}
		\HRule{2.0pt} \\ [0.6cm] \LARGE{Subtitle} \vspace*{10\baselineskip}}
		}
\date{}
\author{\textbf{Author} \\ 
		Terry Qiu \\
		Urbana-Champaign \\
		Fall 2025}

\maketitle
\newpage

\tableofcontents
\newpage

% ------------------------------------------------------------------------------

\section{Integers and Permutations}

\subsection{Integers Modulo $n$}

\begin{definition}{Congruence modulo $n$}
    Let $a,b$ be integers. We say that $a$ is congruent to $b$ modulo $n$ if
    \[n|(a-b) \]
    where $n$ is some integer also. 
\end{definition}
Alternatively, notice that $a$ is congruent to $b$ modulo $n$ also means that 
$a$ has remainder $b$ when divided by $n$, $a=nk+b$.

\vspace{4mm}
Congruence as such defined a equivalence relation on $\mathbb{Z}$. In other words,
\[a\equiv b\pmod{n} \iff n|(a-b)\]
\begin{theorem}
    \[a\equiv b\pmod{n} \iff n|(a-b)\]
    This defines a valid equivalence relation, for $n\geq 2$.
\end{theorem}

\begin{proof}
    It suffices to show that $a\equiv b \pmod{n}$ is reflexive, transitive and symmetric.
    
    \vspace{4mm}
    Notice that for any $a\in \mathbb{Z}$ and any $n\geq 2$, we have 
    \[ n|(a-a)=n|0\]
    So $a\equiv a \pmod{n}$ for any $a$. If $n|(a-b)$ and $n|(b-c)$, then
    \[k_1n=a-b, \quad k_2n=b-c\]
    But then 
    \[n(k_1+k_2)= a-c\]
    So $n|(a-c)$, which implies $a\equiv c\pmod{n}$. For symmetry, notice that 
    \[n|(a-b) \implies k_1n=(a-b) \implies (-k_1)n =(b-a)\]
    So $n|(b-a)$.
    \vspace{4mm}
\end{proof}
\begin{theorem}
    If $a\in \mathbb{Z}$, then $[a]=[r]$ for some $0\leq r\leq n-1$, where $n\geq 2$.
\end{theorem}
\begin{proof}
    If $0\leq a\leq n-1$ then we are done. So, there are really two cases to consider
    $a\geq n$ or $a\leq -1$. Suppose that $a\geq n$, then
    \begin{align*}
        [x]=\{a | a\equiv x \text{ mod }n\}
    \end{align*}
    By the division algorithm, $x=nk+r$ for $0\leq r \leq n-1$. But this means that 
    \begin{align*}
        x\equiv r \text{ mod } n
    \end{align*}
    So in fact 
    \begin{align*}
        [x]=\{a | a\equiv r \text{ mod }n\}=[r]
    \end{align*}
    The same is true when $x\leq -1$.
\end{proof}
\begin{theorem}
$[a]=[b]$ if and only if $a\equiv b \pmod{n}$.
\end{theorem}
What this means is that to prove $a\equiv b \pmod{n}$ for some $n\geq 2$. It suffices 
to prove $[a] =[b]$ in $\mathbb{Z}_n$.
\\\\
For example, suppose we wanted to find the remainder of $4^{119}$ when divided by $7$.
It suffices to find the $0\leq r < 7$ such that $4^{119} \equiv r \pmod{7}$. But for any such $r$, we must have
\begin{align*}
    [4^{119}]= [r]
\end{align*}
We can work out $r$ by working with multiplication in $\mathbb{Z}_7$.\\\\
\textbf{Example:}
Prove that $a$ is divisible by $9$ if and only if the sum of its digits is divisible by $9$.
\begin{proof}
    Let $a$ be some arbitrary integer and consider it as its expansion in base $10$.
    Suppose without loss of generality that
    \begin{align*}
        a=d_nd_{n-1}...d_0
    \end{align*}
    Then
    \begin{align*}
        a=d_n\cdot10^n + ...+ d_0 \cdot 10^0
    \end{align*}
    It suffices to show that
    \begin{align*}
        a \equiv 0 \pmod{9}
    \end{align*}
    But this is true in integer mod $9$ if and only if 
    \begin{align*}
        [a]=[0]
    \end{align*}
    But then
    \begin{align*}
        [a]=[d_n\cdot10^n + ...+ d_0 \cdot 10^0]=[d_n][10^n]+...+[d_0]
    \end{align*}
    Notice that
    \begin{align*}
        10\equiv 1 \pmod{9}
        \implies 10^n \equiv 1^n \equiv 1\pmod{9}
    \end{align*}
    So, $[10^n]=[1]$ for any $n\in\mathbb{N}$. This means that
    \begin{align*}
        [a]=[d_n\cdot10^n + ...+ d_0 \cdot 10^0]=[d_n][1]+...+[d_0]= [d_n]+...+[d_0]
    \end{align*}
    So 
    \begin{align*}
        a \equiv d_n+d_{n-1}+...+d_0 \pmod{9}
    \end{align*}
    Now suppose that $a$ is divisible by $9$, then by transitivity the sum of the digits is divisible by $9$. On the otherhand, if the sum
    of the digits is divisible by $9$, then by transitivity $a$ is divisible by $9$ too. This is what we needed to show.
    
\end{proof}
\begin{definition}
    We call $[0]$ and $[1]$ the roots and unity of $\mathbb{Z}_n$ respectively.
\end{definition}
Suppose we are presented with the problem of solving congruence equations. For example, consider
in integers modulo $17$, the equation
\begin{align*}
    5x\equiv 2 \text{ mod } 17
\end{align*}
We want to find a $x$ such that $5x$ has remainder $2$ when divided by $17$. We can guess the solution by trial an error, but the better way is to 
consider the following problem. We ask whether $5$ has an inverse element.
\begin{definition}
    An inverse element of $[a]$ in the integers modulo $n$ is the class $[b]$ such
    that $[a][b]=[1]$.
\end{definition}
Why does this matter? If $5$ has an inverse then we can multiply both sides by that inverse. Suppose that the inverse 
is $[b]$. We have
\begin{align*}
    [5][b][x]&=[2][b] \\
    [x]&=[2][b]=[2b]
\end{align*}
If an inverse exists, then $[x]=[2b]$. Does there exists a solution to congruence relations of the sort 
\[ax\equiv b \text{ mod } n\]
when $a$ is non invertible? We have a solution if $d=\gcd(a,n)\mid b$ and we dont have a solution when $d=\gcd(a,n)\nmid b$.
We have the following theorem relating to whether an element is invertible in $\mathbb{Z}_n$.
\begin{theorem}
    $[a]$ is invertible in $\mathbb{Z}_n$ if and only if $\gcd(a,n)=1$.
\end{theorem}
\begin{proof}
    If $\gcd(a,n)=1$, then $1=ax+ny$ for some integers $x,y$. This means alternatively that $n|(1-ax)$ which means
    \[[ax]\equiv [1] \text{ mod } n\]
    So $[a]$ is invertible in particular by $[x]$. The reverse is also obvious.
\end{proof}
The following theorem is convenient for solving congruence relations.
\begin{theorem}{Chinese Remainder Theorem or the CRT}
    The equations
    \begin{align*}
        x \equiv s \text{ mod } m \quad x \equiv t \text{ mod } n
    \end{align*}
    has a solution $x$ when $m,n$ are relatively prime.
\end{theorem}
\begin{proof}
    Suppose $m,n$ are relatively prime. We can write $1$ as a linear combination of
    $m$ and $n$ by Bezout's identity.
    \begin{align*}
        1=mx+ny
    \end{align*}
    Let \begin{align*}
        x=mxt+nys
    \end{align*}
    Then
    \begin{align*}
        x-s = mxt+(ny-1)
    \end{align*}
    But $ny-1 =-mx$. So 
    \begin{align*}
        x-s = mxt-mx =mx(t-1)
    \end{align*}
    So $m|(x-s)$. This is what it means for $x$ to be congruent to $s$ modulo $m$.
    Similarly, 
    \begin{align*}
        x-t&=(mx-1)t+nys \\
        x-t &= ny(s-t)
    \end{align*}
    So $n|(x-t)$.
\end{proof}
\newpage
\subsection{Permutations}
\begin{definition}
    A permutation on $X_n=\left\{ 1,2,...,n \right\}$ is a bijective function 
    \[
    \sigma: X_n \rightarrow X_n
    \]
    We write $S_n$ for the group of all permutations on $X_n$.
\end{definition}
\begin{theorem}
    The set $S_n$ of permutations has $n!$ elements.
\end{theorem}
We say that an element is moved by a permutation if it is mapped to another element. Otherwise, if it is mapped to itself, we call it fixed.
Two permutations are disjoint if the intersection of the sets moved by the permutations is the empty set.
\begin{theorem}
If $k$ is moved by $\sigma$ then $\sigma k$ is moved by $\sigma $ also.
\end{theorem}
\begin{proof}
    Suppose $k$ is moved by $\sigma$ but $\sigma k$ is not moved by $\sigma$. Then 
    $\sigma(\sigma k)=\sigma k$. By injectivity, $\sigma k=k$. But $k$ is moved by $\sigma$. This is a contradiction. $\sigma k$ must be moved by $\sigma$ also.
\end{proof}
This means that if we have an element moved by $\sigma$, then permuting it by $n$ times will still give us an element that is moved by $\sigma$. In other words, we cannot go from being moved 
by $\sigma$, to not being moved by $\sigma$.
\begin{theorem}
    If $\sigma$ and $\tau$ are disjoint, then they commute.
    \[
    \sigma \tau = \tau \sigma
    \]
\end{theorem}
\begin{proof}
    It suffices to show that for every element in the set that they biject on $\sigma \tau$ takes on the same 
    value as $\tau \sigma$.
    \\\\
    There are three cases,\\\\
    $\textbf{Case 1}$
    Suppose that $k$ is moved by $\tau$ but not by $\sigma$. Then 
    \[
    \tau\sigma(k)=\tau (k)
    \]
    By the previous result $\tau k$ is still moved by $\tau$. Because $\sigma$ and $\tau$ are disjoint
    $\tau k$ cannot be moved by $\sigma$. So
    \[
    \tau\sigma(k)=\tau (k)= \sigma\tau(k)
    \]
    $\textbf{Case 2}$ 
    Suppose $k$ is moved by $\sigma$ but fixed by $\tau$. This is the symmetric case to Case $1$.
    $\textbf{Case 3}$ 
    Suppose $k$ is moved by neither $\sigma$ nor by $\tau$, then
    \[
    \sigma \tau k = k =\tau \sigma k
    \]
    In all three cases, $\sigma$ and $\tau$ commutes.
\end{proof}
\begin{theorem}
    Every non identity permutation is a product of disjoint cycles of length at least $2$
\end{theorem}
\begin{theorem}
    Every cycle with length $r\geq 2$ can be factored into $r-1$ transpositions.
\end{theorem}
\begin{theorem}
    If a permutation can be factored into $m$ transpositions and $n$ transpositions. Then $m$ and $n$ are of the same parity.
\end{theorem}
\begin{definition}
    We say that a permutation $\sigma$ is odd if it can be factored into an odd number of transpositions. We say
    that it is even otherwise.
\end{definition}
To determine the parity of any permutation, we can first factor them into disjoint cycles. Then, we can use the fact
that an $r$-cycle is a product of $r-1$ transpositions to determine the parity of each cycles individually. Of course,
the parity of the entire group is the sum of the parities of the $r$-cycles. 
\section{Groups}
\subsection{Binary operations}

\subsection{Groups}

\newpage
\subsection{Cyclic Groups}
Let $G$ be a group and let $g$ be an element of $G$. Write
\[
\langle g\rangle =\left\{ g^k| k\in\mathbb{Z} \right\}
\]
\begin{theorem}
    For any $g\in G$, $\langle g \rangle$ is a subgroup of $G$ and for every subgroup $H$ of $G$ containing $g$,
    $\langle g \rangle \subseteq H$
\end{theorem}
\begin{proof}
    The proof for the first statement is clear, so we show only that any subgroup $H$ containing $g$ contain $\langle g \rangle$.
    Since $H$ contains $g$, it contains any finite composition of $g$. In other words, $gg=g^2\in H$. This means that $g^k\in H$ for any $k\in \mathbb{Z}$.
\end{proof}
\begin{definition}
    The order of an element $g$ of a group $G$ is the smallest number $n$ such that 
    \[
    g^n=1
    \]
    If $g^n=1$ only when $n=0$,  then we say that the order of $g$ is infinite.
\end{definition}
The following theorem works with the situation where $g$ has order $n$.
\begin{theorem}
    
    If the order of $g\in G$ is $n$, then $g^k =1$ if and only if $n|k$.
\end{theorem}
\begin{proof}
    Suppose that $n|k$, then $k=qn$ for some $q\in \mathbb{Z}$, write $g^k=g^{qn}=(g^n)^q=1^q=1$. On the other hand,
    suppose that $g^k=1$. Write $k=qn+r$ for some $0\leq r\leq n-1$, then 
    \[
    g^r =g^kg^{-qn}=g^k(g^n)^{-q}=g^k1^{-q}=g^k1=g^k=1
    \]
    But if $r\leq n-1$ and $r$ is not $0$, then $g^r=1$ contradicts the assumption that $g$ has order $n$. So $r$ must be $0$. This completes the proof.
\end{proof}
\begin{theorem}
    $g^k=g^m$ if and only if $k\equiv m \text{ mod } n$.
\end{theorem}
\begin{proof}
    Notice that $g^{k-m}=1$. But by theorem $2.3.2$, $n|k-m$. This is what it means for $k$ to be congruent to $m$.
\end{proof}
\begin{theorem}
    $\langle g \rangle=\left\{ 1, g, g^2,g^3,\cdots, g^{n-1} \right\}$ where $g^i$ is distinct for $0\leq i\leq n-1$.
\end{theorem}
This theorem says that $\langle g \rangle$ is in fact finite if the order of $G$ is finite. This is not entirely obvious, so here we give a proof.
\begin{proof}
    $\langle g \rangle\supseteq\left\{ 1, g, g^2,g^3,\cdots, g^{n-1} \right\}$ is clear so we prove $\langle g \rangle\subseteq\left\{ 1, g, g^2,g^3,\cdots, g^{n-1} \right\}$. 
    Let $x\in \langle g \rangle$ be arbitrary. Then $x=g^k$ for some $k\in \mathbb{Z}$. Write $k=qn+r$ for $0\leq r < n$. Then 
    \[
    g^k = g^{qn+r}=g^{qn}g^r= (g^n)^qg^r= 1^qg^r=g^r
    \]
    So any element in $\langle g \rangle$ is $g^i$ for $0\leq i \leq n-1$. This is what we needed to show. 
    \\\\
    To show that they are distinct, take two elements $g^m$ and $g^k$ from the set. If $g^m =g^k$, then
    \[
    g^{m-k}=1
    \]
    This implies that $n|m-k$. But $0\leq m-k < n$. So $m-k=0$. So they are the same if and only if they have the same power. 
    In otherwords, no elements with distinct powers are the same.
\end{proof}
\begin{theorem}
    If the order of $g$ is $n$ and if $d|n$ with $d\geq 1$, then the order of $g^d$ is $\frac{n}{d}$.
\end{theorem}
\begin{proof}
    Notice that
    \[
    (g^d)^{\frac{n}{d}}=g^n=1
    \]
    where $n$ is the order of $g$. So it is enough to show that $\frac{n}{d}$ is the smallest such power. Suppose there exists
    an $r\geq 1$ such that
    \[
    (g^d)^r=g^{dr}=1
    \]
    then $n|dr$. Write $dr=qn$ for some $q$. Then since $n=d\frac{n}{d}$, we have that $dr =qd\frac{n}{d}$. But $d\geq 1$. So $r=q\frac{n}{d}$. But this means that $\frac{n}{d}$ divides $r$ so 
    that $r$ is at least as big as $\frac{n}{d}$. This shows that $\frac{n}{d}$ is the minimal power such that $g^{\frac{n}{d}}=1$ is the identity.
\end{proof}
\begin{theorem}
    Every cyclic group is abelian. The converse is not generally true.
\end{theorem}
This theorem is a consequence of addition being commutative.
\begin{theorem}
    Any subgroup $H$ of a cyclic group $G = \langle g \rangle$ is cyclic. In fact, $H$ is generated by $g^m$ where $m$ is smallest $m$ such that 
    $g^m$ in $H$.
\end{theorem}
\begin{proof}
    Suppose that $H$ is a subgroup of $G$ and $\langle g \rangle$ is a cyclic group. If $H=\left\{ 1 \right\}$ then $H=\langle 1 \rangle$. Suppose that $H$ 
    contains elements beyond the identity. Then $g^m\in H$ for some $m\in \mathbb{Z}$. Since the inverse of $g^m$ must exists for any $m$, it is enough to consider the cases where $g^m\in H$ for positive $m$s.
    In particular, pick the smallest positive $m$ such that $g^m\in H$. We want to show that $\langle g^m \rangle = H$. Pick any element $h\in H$, 
    so $h=g^k$ for some $k\in \mathbb{N}$. By the Euclidean algorithm there exists $0\leq r\leq k-1$ such that
    \[
    k=mq+r 
    \]
    so $g^k=g^{mq+r}$. But then $g^r =g^kg^{-mq}=g^k(g^{m})^{-q}\in H$ eventhough $r< m$. This means that $r$ must be equal to $0$ otherwise $m$ is not the smallest element. In otherwords, $k$ is an integer 
    multiple of $m$. This means that $g^k\in \langle g \rangle$. This is what we needed to show.

\end{proof}
\begin{theorem}
    Suppose that $G=\langle g \rangle$, then $G=\langle g^k \rangle$ if and only if $\gcd(k,n)=1$ where $n$ is the order of $g$.
\end{theorem}
\begin{proof}
      Suppose that $\gcd(k, n)=1$. Then by Bezout's theorem, there exists solutions $q,t$ to the equation $1=qk+tn$. So let $g\in G$ be arbitrary, notice that 
      \[
      g^1 = g^{qk+tn}= (g^k)^q+(g^n)^t = (g^k)^q \in G
      \]
      So any $g\in G$ is a power of $g^k$. This shows that $G=\langle g^k \rangle$. On the other hand, suppose that 
$G= \langle g^k \rangle$. Let $g\in G$ be abitrary. Then $g=(g^k)^m$ for some $m\in \mathbb{Z}$. But this means that $g^{1-km}=1$. But then $n|1-km$. So $nq =1-km$, such that $nq+km=1$. 
This means that $\gcd(k,n)=1$. Since there exists solutions to the equation $nq+km=1$.
\end{proof}
\begin{theorem}
    Let $G$ be a group with order $n$ and 
    suppose that $H$ is a subgroup of $G$. Then $H=\langle g^d \rangle$ for some $d$ that divides $n$.
\end{theorem}
\begin{proof}
    By theorem $2.3.8$, any subgroup of a cyclic group is cyclic. So $H=\langle g^m \rangle$ for some $m\in \mathbb{Z}$. Let 
    $d=\gcd(m,n)$. Because $H=\langle g^m \rangle$ for some $m$, any element in $H$ is a power of $g^m$. Since $d|m$, $dq=m$ for some $q$. So $g^m= g^{dq}=(g^d)^q\in H$.
    On the other hand, suppose that $x$ is an element $\langle g^d\rangle$, then $x=(g^d)^k$ for some $k \in \mathbb{Z}$ and  
    \[
    (g^d)^k= (g^{qm+tn})^k=g^{qmk}g^{tnk}= (g^{m})^{qk}(g^n)^{tk}\in H
    \]
    THis completes the proof.
\end{proof}
Moreover, we know that if $g$ has order $n$ then $g^d$ must have order $n/d$.
\begin{theorem}
    Suppose that $G=\langle g \rangle$ and the order of $g$ is $n$. If $k|n$, then $\langle g^{\frac{n}{k}} \rangle$ is the unique subgroup of order $k$.
\end{theorem}
\begin{proof}
    Suppose that $K$ is the subgroup of order $k$. Since it is a subgroup there exists some integer $d$ such that 
    \[
    K=\langle g^d \rangle
    \] 
    Additionally, since $d|n$, $dq=n$ for some $q\in \mathbb{Z}$. By theorem $2.3.5$ we have that the order of $g^d$ is $\frac{n}{d}$, but we supposed that
    the orde of $K$ is $k$. So $k=\frac{n}{d}$. In other words, $\frac{n}{k}=d$. This is what we needed to show.
\end{proof}
\newpage
\subsection{Homomorphism and Isomorphisms}
Let $G,H$ be groups. We call a mapping $\phi$ from $G$ to $H$ a homomorphism if 
\[
\phi(g_1g_2)=\phi(g_1)\phi(g_2)
\]
for all $g_1,g_2 \in G$.
\begin{theorem}
    If $\phi$ is a homomorphism from $G$ to $H$ then it has the following properties
    \begin{align*}
        &\phi(e_G)=e_H \\
        &\phi(g^{-1})=(\phi(g))^{-1}\\
        &\phi(g^k)= \phi(g)^k
    \end{align*}
\end{theorem}
\begin{proof}
    For the first property, notice that $\phi(e)=\phi(ee)=\phi(e)\phi(e)$. Left cancelling in $H$ gives 
    $e_H=\phi(e_G)$. For the second property, consider $\phi(g^{-1})\phi(g)=\phi(g^{-1}g)=\phi(e_G)=e_H$. Multiplying 
    both sides on the right by inverse of $\phi(g)$ gives what we want to show.
    \\\\
    For the third property, there are two cases two consider, $k\geq 0$ and $k<0$. We prove $k\geq 0$ by induction on $k$. Notice that 
    if $k=0$, $\phi(g^0)=\phi(e_G)=e_H=\phi(g)^0$. Suppose for all $m< k$, the identity holds. Notice that 
    \[
    \phi(g^k)=\phi(g)\phi(g^{k-1})=\phi(g)\phi(g)^{k-1}=\phi(g)^k
    \]
    Now consider the case of $k<0$. Write $k=-m$, for $m>0$ then 
    \[
    \phi(g^k)=\phi(g^{-m})=\phi((g^{m})^{-1})= \phi(g^m)^{-1}= ((\phi(g))^m)^{-1}=\phi(g)^{-m}
    \]
\end{proof}
The idea of these three corollaries is that the homomorphism preserves the power of an element. If an element $h$ was a $k$th power of $g$, then it will be mapped to
the $k$th power of whatever $g$ is mapped.

% ------------------------------------------------------------------------------
% Reference and Cited Works
% ------------------------------------------------------------------------------

\bibliographystyle{IEEEtran}
\bibliography{References.bib}

% ------------------------------------------------------------------------------

\end{document}